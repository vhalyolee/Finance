\documentclass[12pt]{article}

\topmargin 0.0cm
\oddsidemargin 0.2cm
\textwidth 16cm 
\textheight 21cm
\footskip 1.0cm

\usepackage{amsfonts, amsmath}
\usepackage{times, graphicx}

\newenvironment{sciabstract}{%
\begin{quote} \it}
{\end{quote}}

\title{A Simple SABR Based Aproximation for the Cross Smile Marking}

\author{Nic Hutchings, Valerie Halyo, Almada Sergio}

\begin{document}
\baselineskip24pt
\maketitle

\begin{sciabstract}
      A simple approximation of the  cross smile formulation is described in terms of SABR parameters.
      The resulting cross smile SABR paramters are composed of 2 terms, the first term  solely depends on the diagonal
      SABR model dynamics and the second term  is evaluated by either calibrating to the market cross smile or using
      historical correlation data. Each one of these options correspond to a different use case and strategy that is implemented
      in the GUI. The first case corresponds to liquid cross currency pairs implemented in the marked strategy and the second is described
      in historical correlation strategy.
      

\end{sciabstract}

\section{Model Set Up}
\paragraph*{}
Consider a collection of log-normal SABR models with dynamics,
$$dS_{i,t} = \alpha_{i,t}S_{i,t}dW_{i,t},$$
$$d\alpha_{i,t} = \nu_i\alpha_{i,t}dZ_{i,t},$$
where ${\bf E}[dW_{i,t}dZ_{i,t}] = \rho_idt$ with $i = 1,2,X.$ $\rho_i$ and $\nu_i$ represent the spot-volatility correlation and volatility of volatility respectively within each SABR model. Each dynamic for pair $i$ is written in its respective {\it domestic} measure.

\section{Cross Smile Volatility, $\alpha_X$}
\paragraph*{}
Now consider a cross FX rate defined through, either $S_{X,t} = S_{1,t}S_{2,t}$ or   $S_{X,t} = \frac{S_{1,t}}{S_{2,t}}$,  
We follow the exact same methodology to derive the cross smile SABR parameters in both cases and since the results differ only in a sign flip
in 3 places we describe the derivation for  $S_{X,t} = S_{1,t}S_{2,t}$ and provide a generic form of the final result.

we can write,
$$dS_{X,t} = S_{1,t}dS_{2,t} + S_{2,t}dS_{1,t} + dS_{1,t}dS_{2,t}.$$
Inserting terms and including the drift adjustment required to move to the X-domestic measure,
\begin{equation}
  \begin{split}
    dS_{X,t} &= \alpha_{X,t}S_{X,t}dW_{2,t} + \alpha_{1,t}S_{X,t}dW_{1,t} + \alpha_{1,t}\alpha_{2,t}S_{X,t}\rho_{S_1,S_2}dt - \alpha_{1,t}\alpha_{2,t}S_{X,t}\rho_{S_1,S_2}dt \\
    &= \alpha_{1,t}S_{X,t}dW_{1,t} + \alpha_{2,t}S_{X,t}dW_{2,t}.
  \end{split}
\end{equation}

We know that we also have,
\begin{equation}
  dS_{X,t} = \alpha_{X,t}S_{X,t}dW_{X,t}.
\end{equation}
Equating the variance of both expressions for $dS_{X,t}$, leads to the well-known triangle relationship in FX,
\begin{equation}
  \alpha^2_{X,t} = \alpha^2_{1,t} + \alpha^2_{2,t} \pm 2\rho_{S_1,S_2}\alpha_{1,t}\alpha_{2,t}.
\end{equation}

Here the positive sign is for the $S_{X,t} = S_{1,t}S_{2,t}$ case and the negative sign is for the $S_{X,t} = \frac{S_{1,t}}{S_{2,t}}$ case.

The correlation between spots $S_1$ and $S_2$ $\rho_{S_1,S_2}$ is evaulated using .

\begin{equation}
  \rho_{S_1,S_2} = \pm \frac{\text{atm}^2_{X,t} - \text{atm}^2_{1,t} - \text{atm}^2_{2,t} } { 2\text{atm}_{1,t}\text{atm}_{2,t}}.
\end{equation}

Note for most corss FX pairs the ATM volatility is reasonably observable, which should give a reasonable first approximation to $\alpha_X$.


\section{Cross Smile Volatility of Volatitlity, $\nu_X$}
\paragraph*{}
Now differentiating expression (3) leads to,
\begin{multline}
  2\alpha_{1,t}d\alpha_{1,t} + (d\alpha_{X,t})^2 = 2d\alpha_{1,t}(\alpha_{1,t} + \rho_{S_1,S_2}\alpha_{2,t}) + 2d\alpha_{2,t}(\alpha_{2,t} + \rho_{S_1,S_2}\alpha_{1,t}) \\
  + (d\alpha_{1,t})^2 + (d\alpha_{2,t})^2 + 2\rho_{S_1,S_2}d\alpha_{1,t}d\alpha_{2,t}
\end{multline}

matching drifts on both sides and remembering to incorporate the drift adjustments required to switch to the X-domestic meausre in the pair 2 processes, we have,
\begin{multline}
  \nu^2_X\alpha^2_{X,t} = \nu^2_1\alpha^2_{1,t} + \nu^2_2\alpha^2_{2,t} + 2\nu_1\nu_2\alpha_{1,t}\alpha_{2,t}\rho_{S_1,S_2}\rho_{\alpha_{1,t},\alpha_{2,t}} - 2\nu_2\alpha_{1,t}\rho_{S_1,S_2}\alpha_{2,t}(\alpha_{2,t} + \rho_{S_1,S_2}\alpha_{1,t}), \\ \\
  \nu^2_X = \frac{\nu^2_1\alpha^2_{1,t} + \nu^2_2\alpha^2_{2,t}}{\alpha^2_{X,t}} +
  \frac{2\alpha_{1,t}\alpha_{2,t}\rho_{S_1,S_2}\nu_2(\pm\nu_1\rho_{\alpha_{1,t},\alpha_{2,t}} - \rho_2(\alpha_{2,t} \pm \rho_{S_1,S_2}\alpha_{1,t})}{\alpha^2_{X,t}}
\end{multline}

We have derived an expression for the cross volatility of volatility in terms of parameters calibrated to the diagonal smiles and an extra term
that could be evaluated by calibrating it to the cross market smile  or by extracting historical correlation data.


%correlation parameter $\rho_{\alpha_{1,t},\alpha_{2,t}}$, the correlation between the two diagonal volatility processes and the cross colatility $\alpha_{X,t}$. It would seem reasonable to assume a high value for the volatility-volatility correlation. 

\section{Cross Smile Volatility-Spot correlation , $\rho_X$}

To complete the cross smile paramter set we first rewrite equation(4) as

\begin{equation}
  \frac{d \alpha_{X,t}}{\alpha_{X,t}} = \omega_{+}(\alpha_{1,t},\alpha_{2,t},\alpha_{X,t}) \frac{d \alpha_{1,t}}{\alpha_{1,t}} + \omega_{-}(\alpha_{1,t},\alpha_{2,t},\alpha_{X,t}) \frac{d \alpha_{2,t}}{\alpha_{2,t}} + (.)dt
\end{equation}

where the functions $\omega_{\pm}(.)$ are given by ,

\begin{equation}
\omega_{\pm} (\alpha_{1,t},\alpha_{2,t},\alpha_{X,t})  = \frac{1}{2} (1 \pm \frac{\alpha^2_{1,t} - \alpha^2_{2,t}}{\alpha^2_{X,t}})
\end{equation}

Evaluating the expression for $Cov[\frac{dS_X}{S_X},\frac{d \alpha_{X,t}}{\alpha_{X,t}}]$

\begin{equation}
Cov[\frac{dS_X}{S_X},\frac{d \alpha_{X,t}}{\alpha_{X,t}}] = Cov[\frac{dS_1}{S_1}+\frac{dS_2}{S_2},\omega_{+}(\alpha_{1,t},\alpha_{2,t},\alpha_{X,t}) \frac{d \alpha_{1,t}}{\alpha_{1,t}} + \omega_{-}(\alpha_{1,t},\alpha_{2,t},\alpha_{X,t}) \frac{d \alpha_{2,t}}{\alpha_{2,t}}].
\end{equation}

We make the approximation of assuming the $\omega_{\pm}(.)$ are constant, which lead to 

\begin{multline}
  \rho_{X}  \approx \frac{\rho_{1} \nu_{1} \alpha_{1} \omega_{+}(\alpha_{1,t},\alpha_{2,t},\alpha_{X,t}) +\rho_{2} \nu_{2} \alpha_{2} \omega_{-}(\alpha_{1,t},\alpha_{2,t},\alpha_{X,t}) }{\alpha_{X} \nu_{X}}  \\   + \frac{ \rho_{1} \nu_{1} \alpha_{1} \omega_{+}(\alpha_{1,t},\alpha_{2,t},\alpha_{X,t}) +  \rho_{2} \nu_{2} \alpha_{2} \omega_{-}(\alpha_{1,t},\alpha_{2,t},\alpha_{X,t}) }{\alpha_{X} \nu_{X}}.
\end{multline}

\section{Resultant Cross Smile SABR Parametrization }


\begin{alignat}{3}
 &\nu^2_X && \approx & \frac{\nu^2_1\alpha^2_{1,t} + \nu^2_2\alpha^2_{2,t}}{\alpha^2_{X,t}} \qquad \qquad \qquad \qquad \qquad \qquad \qquad \qquad  + \Delta_{\nu} &, \\
 &\rho_{X} && \approx &\qquad  \frac{\rho_{1} \nu_{1} \alpha_{1} \omega_{+}(\alpha_{1,t},\alpha_{2,t},\alpha_{X,t}) +\rho_{2} \nu_{2} \alpha_{2} \omega_{-}(\alpha_{1,t},\alpha_{2,t},\alpha_{X,t}) }{\alpha_{X} \nu_{X}}  + \Delta_{\rho} &
\end{alignat}




\end{document}
